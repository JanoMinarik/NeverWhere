\chapter{Hartree-Fock}

\section{Basis Function}
For small symmetric systems, such as atoms, the Hartree-Fock equations may be solved by mapping the orbitals on a set of grid points. These are referred to as numerical Hartree-Fock methods \cite{jensen07}. But, almost all calculations use a basis set epansion to express the unknown MOs in terms of a set of known functions. Any type of a function may be used, such as exponential, Gaussian, polynomial or wavelets. When choosing the basis function there are two aspects to mind. First is the physics of the problem, second is calculation complexity.

% start of equations
Basically each MO $\phi$ is expanded in terms of the basis functions $\chi$. Conventionally as a linear combination of atomic orbitals (LCAO).
\begin{equation}
    \phi_i = \sum_{\alpha}^{N_{basis}}c_{\alpha,i}\chi_{\alpha}
\end{equation} 
